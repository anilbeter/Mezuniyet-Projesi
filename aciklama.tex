\documentclass{article}
\usepackage[utf8]{inputenc}
\usepackage[T1]{fontenc}
\usepackage{amsmath} % Denklemler için
\usepackage{amssymb} % \checkmark ve \times sembolleri için

\begin{document}

\section*{Kural 2'nin Mantıksal Açıklaması (Kör Nokta Kalmayacak!)}

``Kapsanması zorunlu'' denilen \textit{her bir rotanın} istisnasız bir şekilde denetlenmesini garanti altına alır.

\subsection*{1. Malzemeler (Senin Verilerin)}

\begin{itemize}
  \item \textbf{Rotalar ($I$ Kümesi):} Kapsamak istediğin 2 rota var:
        \begin{itemize}
          \item $i_1$: Ankara $\rightarrow$ İzmir Rotası
          \item $i_2$: Bolu $\rightarrow$ İzmir Rotası
        \end{itemize}

  \item \textbf{Konumlar ($J$ Kümesi):} Sensör koyabileceğin 3 yer var:
        \begin{itemize}
          \item $j_1$: Gerede
          \item $j_2$: Susurluk
          \item $j_3$: Manisa
        \end{itemize}

  \item \textbf{$a_{ij}$ Matrisi (Kim Kimi Görüyor?):}
        \begin{itemize}
          \item $i_1$ (Ankara $\rightarrow$ İz) rotası: Gerede ($j_1$), Susurluk ($j_2$) ve Manisa'dan ($j_3$) geçer.
          \item $i_2$ (Bolu $\rightarrow$ İz) rotası: Gerede'den ($j_1$) \textbf{GEÇMEZ}, ama Susurluk ($j_2$) ve Manisa'dan ($j_3$) geçer.
        \end{itemize}
\end{itemize}

\subsubsection*{KURAL A (Rota $i_1$ için):}

Model, $i_1$ (Ankara $\rightarrow$ İzmir) rotasını alır ve ``Bu rotayı hangi konumlar görüyordu?'' diye $a_{ij}$ matrisine bakar.
\begin{itemize}
  \item Gerede ($j_1$) görüyor mu? Evet ($a_{1,1}=1$)
  \item Susurluk ($j_2$) görüyor mu? Evet ($a_{1,2}=1$)
  \item Manisa ($j_3$) görüyor mu? Evet ($a_{1,3}=1$)
\end{itemize}

Şimdi denklemi yazar:
\begin{equation*}
  (a_{1,1} \cdot x_1) + (a_{1,2} \cdot x_2) + (a_{1,3} \cdot x_3) \ge 1
\end{equation*}

Yani:
\begin{equation*}
  (1 \cdot x_1) + (1 \cdot x_2) + (1 \cdot x_3) \ge 1
\end{equation*}

\begin{itemize}
  \item \textbf{Bu Denklemin Anlamı:} ``Ankara-İzmir rotasını kurtarmak için, Gerede'ye sensör kurma kararın ($x_1$), Susurluk'a kurma kararın ($x_2$) veya Manisa'ya kurma kararın ($x_3$)\dots Bu üç karardan \textbf{en az bir tanesi 'EVET' (yani 1)} olmalıdır.''
  \item Eğer üçüne de kurmazsan ($x_1=0, x_2=0, x_3=0$), toplam '0' olur. $0 \ge 1$ denklemi YANLIŞ çıkar ve model bu çözümü reddeder.
\end{itemize}

\subsubsection*{KURAL B (Rota $i_2$ için):}

Model, $i_2$ (Bolu $\rightarrow$ İzmir) rotasını alır ve ``Bu rotayı hangi konumlar görüyordu?'' diye matrise bakar.
\begin{itemize}
  \item Gerede ($j_1$) görüyor mu? Hayır ($a_{2,1}=0$)
  \item Susurluk ($j_2$) görüyor mu? Evet ($a_{2,2}=1$)
  \item Manisa ($j_3$) görüyor mu? Evet ($a_{2,3}=1$)
\end{itemize}

Şimdi ikinci denklemi yazar:
\begin{equation*}
  (a_{2,1} \cdot x_1) + (a_{2,2} \cdot x_2) + (a_{2,3} \cdot x_3) \ge 1
\end{equation*}

Yani:
\begin{equation*}
  (0 \cdot x_1) + (1 \cdot x_2) + (1 \cdot x_3) \ge 1
\end{equation*}

\begin{itemize}
  \item \textbf{Bu Denklemin Anlamı:} ``Bolu-İzmir rotasını kurtarmak için, Susurluk'a kurma kararın ($x_2$) veya Manisa'ya kurma kararın ($x_3$)\dots Bu iki karardan \textbf{en az bir tanesi 'EVET' (yani 1)} olmalıdır.''
  \item \textbf{Kritik Detay:} Model burada Gerede'ye sensör kurup kurmamanı ($x_1$) \textbf{umursamaz}. Çünkü $a_{2,1}=0$. Yani sen Gerede'ye sensör kursan bile ($x_1=1$ olsa dahi), $0 \cdot 1 = 0$ olacağı için bu rotanın denetlenmesine zerre faydası olmaz.
\end{itemize}

\subsection*{3. Modelin Amacıyla Çatışması (Çözüm)}

Modelin iki zıt güdüsü vardır:
\begin{enumerate}
  \item \textbf{Kural 1 (Amaç):} ``Maliyeti düşür!'' (Yani $x_1, x_2, x_3$'ü mümkünse 0 yap).
  \item \textbf{Kural 2 (Kısıtlar):} ``Kör nokta bırakma!'' (Yani aşağıdaki iki kuralı da SAĞLA):
        \begin{itemize}
          \item Kural A: $x_1 + x_2 + x_3 \ge 1$ (Rota 1'i kurtar)
          \item Kural B: $x_2 + x_3 \ge 1$ (Rota 2'yi kurtar)
        \end{itemize}
\end{enumerate}

Model bu iki zıt güdüyü dengeler. ``Hem Kural A'yı hem de Kural B'yi aynı anda sağlayan \textbf{en az sayıda} $x$ (sensör) hangisidir?'' diye bakar.

\begin{itemize}
  \item ``Sadece Gerede'ye kursam ($x_1=1$)?''
        \begin{itemize}
          \item Kural A: $1+0+0 \ge 1$ (Sağlandı $\checkmark$)
          \item Kural B: $0+0 \ge 1$ (Sağlanmadı! $\times$) $\rightarrow$ Geçersiz Çözüm.
        \end{itemize}

  \item ``Sadece Susurluk'a kursam ($x_2=1$)?''
        \begin{itemize}
          \item Kural A: $0+1+0 \ge 1$ (Sağlandı $\checkmark$)
          \item Kural B: $1+0 \ge 1$ (Sağlandı $\checkmark$) $\rightarrow$ \textbf{Geçerli Çözüm!}
        \end{itemize}
\end{itemize}

Model, sadece Susurluk'a ($x_2=1$) bir sensör kurarak \textit{tüm rotaları} (hem $i_1$'i hem $i_2$'yi) denetleyebileceğini bulur. Eğer Manisa'nın maliyeti Susurluk'tan daha ucuzsa, cevabı ``Manisa'ya kur'' olarak da verebilir. Ama sonuçta 1 sensörle bu işi çözebileceğini anlar.

\bigskip % Biraz boşluk ekler
\textbf{Özetle Kural 2:} Belirlediğin her bir rotanın denetimsiz kalmamasını (en az bir sensör tarafından görülmesini) matematiksel olarak garanti altına alan ``sigorta'' kurallarıdır.

\end{document}